\PassOptionsToPackage{unicode=true}{hyperref} % options for packages loaded elsewhere
\PassOptionsToPackage{hyphens}{url}
\PassOptionsToPackage{dvipsnames,svgnames*,x11names*}{xcolor}
%
\documentclass[a4paper,]{report}
\usepackage{lmodern}
\usepackage{amssymb,amsmath}
\usepackage{ifxetex,ifluatex}
\usepackage{fixltx2e} % provides \textsubscript
\ifnum 0\ifxetex 1\fi\ifluatex 1\fi=0 % if pdftex
  \usepackage[T1]{fontenc}
  \usepackage[utf8]{inputenc}
  \usepackage{textcomp} % provides euro and other symbols
\else % if luatex or xelatex
  \usepackage{unicode-math}
  \defaultfontfeatures{Ligatures=TeX,Scale=MatchLowercase}
\fi
% use upquote if available, for straight quotes in verbatim environments
\IfFileExists{upquote.sty}{\usepackage{upquote}}{}
% use microtype if available
\IfFileExists{microtype.sty}{%
\usepackage[]{microtype}
\UseMicrotypeSet[protrusion]{basicmath} % disable protrusion for tt fonts
}{}
\IfFileExists{parskip.sty}{%
\usepackage{parskip}
}{% else
\setlength{\parindent}{0pt}
\setlength{\parskip}{6pt plus 2pt minus 1pt}
}
\usepackage{xcolor}
\usepackage{hyperref}
\hypersetup{
            pdftitle={Low-Level Software Security for Compiler Developers},
            colorlinks=true,
            linkcolor=Maroon,
            filecolor=Maroon,
            citecolor=Blue,
            urlcolor=Blue,
            breaklinks=true}
\urlstyle{same}  % don't use monospace font for urls
\usepackage{graphicx,grffile}
\makeatletter
\def\maxwidth{\ifdim\Gin@nat@width>\linewidth\linewidth\else\Gin@nat@width\fi}
\def\maxheight{\ifdim\Gin@nat@height>\textheight\textheight\else\Gin@nat@height\fi}
\makeatother
% Scale images if necessary, so that they will not overflow the page
% margins by default, and it is still possible to overwrite the defaults
% using explicit options in \includegraphics[width, height, ...]{}
\setkeys{Gin}{width=\maxwidth,height=\maxheight,keepaspectratio}
\setlength{\emergencystretch}{3em}  % prevent overfull lines
\providecommand{\tightlist}{%
  \setlength{\itemsep}{0pt}\setlength{\parskip}{0pt}}
\setcounter{secnumdepth}{5}
% Redefines (sub)paragraphs to behave more like sections
\ifx\paragraph\undefined\else
\let\oldparagraph\paragraph
\renewcommand{\paragraph}[1]{\oldparagraph{#1}\mbox{}}
\fi
\ifx\subparagraph\undefined\else
\let\oldsubparagraph\subparagraph
\renewcommand{\subparagraph}[1]{\oldsubparagraph{#1}\mbox{}}
\fi

% set default figure placement to htbp
\makeatletter
\def\fps@figure{htbp}
\makeatother

\usepackage{makeidx}
\makeindex
\newcounter{TodoCounter}
\usepackage[backgroundcolor=white,linecolor=black]{todonotes}
\let\oldtodo\todo
\usepackage{bclogo}%  \bcpanchant
\renewcommand{\todo}[1]{
  \stepcounter{TodoCounter}
  \oldtodo[caption={\arabic{TodoCounter}. #1}]{\bcpanchant #1}
}
\newcommand{\missingcontent}[1]{
  \stepcounter{TodoCounter}
  \oldtodo[inline,caption={\arabic{TodoCounter}. #1}]{\bcpanchant \textit{#1}}
}

%%%% pandoc-fignos: required package
\usepackage{caption}

%% pandoc-fignos: environment to disable figure caption prefixes
\makeatletter
\newcounter{figno}
\newenvironment{fignos:no-prefix-figure-caption}{
  \caption@ifcompatibility{}{
    \let\oldthefigure\thefigure
    \let\oldtheHfigure\theHfigure
    \renewcommand{\thefigure}{figno:\thefigno}
    \renewcommand{\theHfigure}{figno:\thefigno}
    \stepcounter{figno}
    \captionsetup{labelformat=empty}
  }
}{
  \caption@ifcompatibility{}{
    \captionsetup{labelformat=default}
    \let\thefigure\oldthefigure
    \let\theHfigure\oldtheHfigure
    \addtocounter{figure}{-1}
  }
}
\makeatother

\title{Low-Level Software Security for Compiler Developers}
\date{}

\begin{document}
\maketitle

\clearpage

\vspace*{\fill}
%\includegraphics[height=2cm]{example-image}
This work is licensed under the Creative Commons Attribution 4.0 International
License. To view a copy of this license, visit
http://creativecommons.org/licenses/by/4.0/ or send a letter to Creative
Commons, PO Box 1866, Mountain View, CA 94042, USA.

  © 2021 Arm Limited
  \href{mailto:kristof.beyls@arm.com}{\nolinkurl{kristof.beyls@arm.com}}\\

Version 0-70-g26e4132
\clearpage

{
\hypersetup{linkcolor=}
\setcounter{tocdepth}{2}
\tableofcontents
}
\hypertarget{introduction}{%
\chapter{Introduction}\label{introduction}}

Compilers, assemblers and similar tools generate all the binary code
that processors execute. It is no surprise then that for security
analysis and hardening relevant for binary code, these tools have a
major role to play. Often the only practical way to protect all binaries
with a particular security hardening method is to let the compiler adapt
its automatic code generation.

With software security becoming even more important in recent years, it
is no surprise to see an ever increasing variety of security hardening
features and mitigations against vulnerabilities implemented in
compilers.

Indeed, compared to a few decades ago, today's compiler developer is
much more likely to work on security features, at least some of their
time.

Furthermore, with the ever-expanding range of techniques implemented, it
has become very hard to gain a basic understanding of all security
features implemented in typical compilers.

This poses a practical problem: compiler developers must be able to work
on security hardening features, yet it is hard to gain a good basic
understanding of such compiler features.

This book aims to help developers of code generation tools such as JITs,
compilers, linkers and assemblers to overcome this.

There is a lot of material that can be found explaining individual
vulnerabilities or attack vectors. There are also lots of presentations
explaining specific exploits. But there seems to be a limited set of
material that gives a structured overview of all vulnerabilities and
exploits for which a code generator could play a role in protecting
against them.

This book aims to provide such a structured, broad overview. It does not
necessarily go into full details. Instead it aims to give a thorough
description of all relevant high-level aspects of attacks,
vulnerabilities, mitigations and hardening techniques. For further
details, this book provides pointers to material with more details on
specific techniques.

The purpose of this book is to serve as a guide to every compiler
developer that needs to learn about software security relevant to
compilers. Even though the focus is on compiler developers, we expect
that this book will also be useful to other people working on low-level
software.

\hypertarget{why-an-open-source-book}{%
\section{Why an open source book?}\label{why-an-open-source-book}}

The idea for this book emerged out of a frustration of not finding a
good overview on this topic. Kristof Beyls and Georgia Kouveli, both
compiler engineers working on security features, wished a book like this
would exist. After not finding such a book, they decided to try and
write one themselves. They immediately realized that they do not have
all necessary expertise themselves to complete such a daunting task. So
they decided to try and create this book in an open source style,
seeking contributions from many experts.

As you read this, the book remains unfinished. This book may well never
be finished, as new vulnerabilities continue to be discovered regularly.
Our hope is that developing the book as an open source project will
allow for it to continue to evolve and improve. The open source
development process of this book increases the likelihood that it
remains relevant as new vulnerabilities and mitigations emerge.

Kristof and Georgia, the initial authors, are far from experts on all
possible vulnerabilities. So what is the plan to get high quality
content to cover all relevant topics? It is two-fold.

First, by studying specific topics, they hope to gain enough knowledge
to write up a good summary for this book.

Second, they very much invite and welcome contributions. If you're
interested in potentially contributing content, please go to the home
location for the open source project at
\url{https://github.com/llsoftsec/llsoftsecbook}.

As a reader, you can also contribute to making this book better. We
highly encourage feedback, both positive and constructive criticisms. We
prefer feedback to be received through
\url{https://github.com/llsoftsec/llsoftsecbook}.

\missingcontent{Add section describing the structure of the rest of the book.}

\hypertarget{memory-vulnerability-based-attacks-and-mitigations}{%
\chapter{Memory vulnerability based attacks and
mitigations}\label{memory-vulnerability-based-attacks-and-mitigations}}

\hypertarget{a-bit-of-background-on-memory-vulnerabilities}{%
\section{A bit of background on memory
vulnerabilities}\label{a-bit-of-background-on-memory-vulnerabilities}}

Memory access errors describe memory accesses that, although permitted
by a program, were not intended by the programmer. These types of errors
are usually defined (Hicks \protect\hyperlink{ref-Hicks2014}{2014}) by
explicitly listing their types, which include:

\begin{itemize}
\tightlist
\item
  buffer overflow
\item
  null pointer dereference
\item
  use after free
\item
  use of uninitialized memory
\item
  illegal free
\end{itemize}

Memory vulnerabilities are an important class of vulnerabilities that
arise due to these types of errors, and they most commonly occur due to
programming mistakes when using languages such as C/C++. These languages
do not provide mechanisms to protect against memory access errors by
default. An attacker can exploit such vulnerabilities to leak sensitive
data or overwrite critical memory locations and gain control of the
vulnerable program.

Memory vulnerabilities have a long history. The
\href{https://en.wikipedia.org/wiki/Morris_worm}{Morris worm} in 1988
was the first widely publicized attack exploiting a buffer overflow.
Later, in the mid-90s, a few famous write-ups describing buffer
overflows appeared (Aleph One
\protect\hyperlink{ref-AlephOne1996}{1996}).
\protect\hyperlink{stack-buffer-overflows}{Stack buffer overflows} were
mitigated with \protect\hyperlink{stack-buffer-overflows}{stack
canaries} and \protect\hyperlink{stack-buffer-overflows}{non-executable
stacks}. The answer was more ingenious ways to bypass these mitigations:
\protect\hyperlink{code-reuse-attacks}{code reuse attacks}, starting
with attacks like
\protect\hyperlink{code-reuse-attacks}{return-into-libc} (Solar Designer
\protect\hyperlink{ref-Solar1997}{1997}). Code reuse attacks later
evolved to \protect\hyperlink{code-reuse-attacks}{Return-Oriented
Programming (ROP)} (Shacham \protect\hyperlink{ref-Shacham2007}{2007})
and even more complex techniques.

To defend against code reuse attacks, the
\protect\hyperlink{code-reuse-attacks}{Address Space Layout
Randomization (ASLR)} and
\protect\hyperlink{code-reuse-attacks}{Control-Flow Integrity (CFI)}
measures were introduced. \todo{Refine section
links used here and in the previous paragraph.} This interaction between
offensive and defensive security research has been essential to
improving security, and continues to this day. Each newly deployed
mitigation results in attempts, often successful, to bypass it, or in
alternative, more complex exploitation techniques, and even tools to
automate them.

Memory safe (Hicks \protect\hyperlink{ref-Hicks2014}{2014}) languages
are designed with prevention of such vulnerabilities in mind and use
techniques such as bounds checking and automatic memory management. If
these languages promise to eliminate memory vulnerabilities, why are we
still discussing this topic?

On the one hand, C and C++ remain very popular languages, particular in
the implementation of low-level software. On the other hand, programs
written in memory safe languages can themselves be vulnerable to memory
errors as a result of bugs in how they are implemented, e.g.~a bug in
their compiler. Can we fix the problem by also using memory safe
languages for the compiler and runtime implementation? Even if that were
as simple as it sounds, unfortunately there are types of programming
errors that these languages cannot protect against. For example, a
logical error in the implementation of a compiler or runtime for a
memory safe language can lead to a memory access error not being
detected. We will see examples of such logic errors in compiler
optimizations in a
\protect\hyperlink{jit-compiler-vulnerabilities}{later section}.

Given the rich history of memory vulnerabilities and mitigations and the
active developments in this area, compiler developers are likely to
encounter some of these issues over the course of their careers. This
chapter aims to serve as an introduction to this area. We start with a
discussion of exploitation primitives, which can be useful when
analyzing threat models \todo{Discuss
threat models elsewhere in book and refer to that section here}. We then
continue with a more detailed discussion of the various types of
vulnerabilities, along with their mitigations, presented in a rough
chronological order of their appearance, and, therefore, complexity.

\hypertarget{exploitation-primitives}{%
\section{Exploitation primitives}\label{exploitation-primitives}}

Newcomers to the area of software security may find themselves lost in
many blog posts and other publications describing specific memory
vulnerabilities and how to exploit them. Two very common, yet unfamiliar
to a newcomer, terms that appear in such publications are \emph{read
primitive} and \emph{write primitive}. In order to understand memory
vulnerabilities and be able to design effective mitigations, it's
important to understand what these terms mean, how these primitives
could be obtained by an attacker, and how they can be used.

An \emph{exploit primitive}\index{exploit primitive} is a mechanism that
allows an attacker to perform a specific operation in the memory space
of the victim program. This is done by providing specially crafted input
to the victim program.

A \emph{write primitive}\index{write primitive} gives the attacker some
level of write access to the victim's memory space. The value written
and the address written to may be controlled by the attacker to various
degrees. The primitive, for example, may allow:

\begin{itemize}
\tightlist
\item
  writing a fixed value to an attacker-controlled address, or
\item
  writing to an address consisting of a fixed base and an
  attacker-controlled offset limited to a specific range (e.g.~a 32-bit
  offset)\todo{Consider
  describing in more detail why the range limitation matters}, or
\item
  writing to an attacker-controlled base address with a fixed offset.
\end{itemize}

Primitives can be further classified according to more detailed
properties. See slide 11 of (Miller,
\protect\hyperlink{ref-Miller2012}{n.d.}) for an example.

The most powerful version of a write primitive is an \emph{arbitrary
write} primitive, where both the address and the value are fully
controlled by the attacker.

A \emph{read primitive}\index{read primitive}, respectively, gives the
attacker read access to the victim's memory space. The address of the
memory location accessed will be controlled by the attacker to some
degree, as for the write primitive. A particularly useful primitive is
an \emph{arbitrary read} primitive, in which the address is fully
controlled by the attacker.

The effects of a write primitive are perhaps easier to understand, as it
has obvious side-effects: a value is written to the victim program's
memory. But how can an attacker observe the result of a read primitive?

This depends on whether the attack is interactive or non-interactive (Hu
et al. \protect\hyperlink{ref-Hu2016}{2016}).

\begin{itemize}
\tightlist
\item
  In an \emph{interactive attack}\index{interactive attack}, the
  attacker gives malicious input to the victim program. The malicious
  input causes the victim program to perform the read the attacker
  instructed it to, and to output the results of that read. This output
  could be any kind of output, for example a network packet that the
  victim transmits. The attacker can observe the result of the read
  primitive by looking at this output, for example parsing this network
  packet. This process then repeats: the attacker sends more malicious
  input to the victim, observes the output and prepares the next input.
  You can see an example of this type of attack in (Beer
  \protect\hyperlink{ref-Beer2020}{2020}), which describes a zero-click
  radio proximity exploit.
\item
  In a \emph{non-interactive (one-shot)
  attack}\index{non-interactive (one-shot)
  attack}, the attacker provides all malicious input to the victim
  program at once. The malicious input triggers multiple primitives one
  after the other, and the primitives are able to observe the effects of
  the preceding operations through the victim program's state. The input
  could be, for example, in the form of a JavaScript program (Groß
  \protect\hyperlink{ref-Grouxdf2020}{2020}), or a PDF file pretending
  to be a GIF (Beer and Groß \protect\hyperlink{ref-Beer2021}{2021}).
\end{itemize}

\todo{The references in this section describe complicated modern exploits.
Consider linking to simpler exploits, as well as some tutorial-level material.}

How does an attacker obtain these kinds of primitives in the first
place? The details vary, and in some cases it takes a combination of
many techniques, some of which are out of scope for this book. But we
will be describing a few of them in this chapter. For example a stack
buffer overflow results in a (restricted) write primitive when the input
size exceeds what the program expected.

As part of an attack, the attacker will want to execute each primitive
more than once, since a single read or write operation will rarely be
enough to achieve their end goal (more on this later). How can
primitives be combined to perform multiple reads/writes?

In the case of an interactive attack, preparing and sending input to the
victim program and parsing the output of the victim program are usually
done in an external program that drives the exploit. The attacker is
free to use a programming language of their choice, as long as they can
interact with the victim program in it. Let's assume, for example, an
exploit program in C, communicating with the victim program over TCP. In
this case, the primitives are abstracted into C functions, which prepare
and send packets to the victim, and parse the victim's responses. Using
the primitives is then as simple as calling these functions. These calls
can be easily combined with arbitrary computations, all written in C, to
form the exploit.

For this cycle of repeated input/output interactions to work, the state
of the victim program must not be lost between the different iterations
of providing input and observing output. In other words, the victim
process must not be restarted.

It's interesting to note that while the read/write primitives consist of
carefully constructed inputs to the victim program, the attacker can
view these inputs as \emph{instructions} to the victim program. The
victim program effectively implements an interpreter unintentionally,
and the attacker can send instructions to this interpreter. This is
explored further in (Dullien \protect\hyperlink{ref-Dullien2020}{2020}).

In the case of a non-interactive attack, all computation happens within
the victim program. The duality of input data and code is even more
obvious in this case, as the malicious input to the victim can be viewed
as the exploit code. There are cases for which the input is obviously
interpreted as code by the victim application as well, as in the case of
a JavaScript program given as input to a JavaScript engine. In this
case, the read/write primitives would be written as JavaScript
functions, which when called have the unintended side-effect of
accessing arbitrary memory that a JavaScript program is not supposed to
have access to. The primitives can be chained together with arbitrary
computations, also expressed in JavaScript.

There are, however, cases where the correspondence between data and code
isn't as obvious. For example, in (Beer and Groß
\protect\hyperlink{ref-Beer2021}{2021}), the malicious input consists of
a PDF file, masquerading as a GIF. Due to an integer overflow bug in the
PDF decoder, the malicious input leads to an unbounded buffer access,
therefore to an arbitrary read/write primitive. In the case of
JavaScript engine exploitation, the attacker would normally be able to
use JavaScript operations and perform arbitrary computations, making
exploitation more straightforward. In this case, there are no scripting
capabilities officially supported. The attackers, however, take
advantage of the compression format intricacies to implement a small
computer architecture, in thousands of simple commands to the decoder.
In this way, they effectively \emph{introduce} scripting capabilities
and are able to express their exploit as a program to this architecture.

So far, we have described read/write primitives. We have also discussed
how an attacker might perform arbitrary computations: * in an external
program in the case of interactive attacks, or * by using scripting
capabilities (whether originally supported or introduced by the
attacker) in non-interactive attacks. Assuming an attacker has gained
these capabilities, how can they use them to achieve their goals?

The ultimate goal of an attacker may vary: it may be, among other
things, getting access to a system, leaking sensitive information or
bringing down a service. Frequently, a first step towards these wider
goals is arbitrary code execution\index{arbitrary code execution} within
the victim process. We have already mentioned that the attacker will
typically have arbitrary computation capabilities at this point, but
arbitrary code execution also involves things like calling arbitrary
library functions and performing system calls.

Some examples of how the attacker may use the obtained primitives:

\begin{itemize}
\tightlist
\item
  Leak information, such as pointers to specific data structures or
  code, or the stack pointer.
\item
  Overwrite the stack contents, e.g.~to perform a
  \protect\hyperlink{code-reuse-attacks}{ROP attack}.
\item
  Overwrite non-control data, e.g.~authorization state. Sometimes this
  step is sufficient to achieve the attacker's goal, bypassing the need
  for arbitrary code execution.
\end{itemize}

Once arbitrary code execution is achieved, the attacker may need to
exploit additional vulnerabilities in order to escape a process sandbox,
escalate privilege, etc. Such vulnerability chaining is common, but for
the purposes of this chapter we will focus on:

\begin{itemize}
\tightlist
\item
  Preventing memory vulnerabilities in the first place, thus stopping
  the attacker from obtaining powerful read/write primitives.
\item
  Mitigating the effects of read/write primitives, e.g.~with mechanisms
  to maintain \protect\hyperlink{code-reuse-attacks}{Control-Flow
  Integrity (CFI)}.
\end{itemize}

\hypertarget{stack-buffer-overflows}{%
\section{Stack buffer overflows}\label{stack-buffer-overflows}}

\missingcontent{Describe stack buffer overflows and mitigations}

\hypertarget{code-reuse-attacks}{%
\section{Code reuse attacks}\label{code-reuse-attacks}}

\missingcontent{Discuss ROP, JOP, COOP and mitigations (ASLR, CFI etc)}

\hypertarget{non-control-data-exploits}{%
\section{Non-control data exploits}\label{non-control-data-exploits}}

\missingcontent{Discuss data-oriented programming and other attacks}

\hypertarget{hardware-support-for-protection-against-memory-vulnerabilities}{%
\section{Hardware support for protection against memory
vulnerabilities}\label{hardware-support-for-protection-against-memory-vulnerabilities}}

\missingcontent{Describe architectural features for mitigating memory vulnerabilities and for CFI}

\hypertarget{other-issues}{%
\section{Other issues}\label{other-issues}}

\missingcontent{Mention other issues, e.g. sigreturn-oriented programming}

\hypertarget{jit-compiler-vulnerabilities}{%
\section{JIT compiler
vulnerabilities}\label{jit-compiler-vulnerabilities}}

\missingcontent{Write section on JIT compiler vulnerabilities}

\hypertarget{covert-channels-and-side-channels}{%
\chapter{Covert channels and
side-channels}\label{covert-channels-and-side-channels}}

A large class of attacks make use of so-called side-channels, which are
defined below. The class is so big that in this book we devote the next
two chapters to such attacks. Side-channels have enough complexity to
discuss them separately in this chapter. This chapter describes the
mechanisms used to make communication happen through side-channels. The
next two chapters explore how attacks are constructed that use
side-channels.

Side-channels and covert channels are closely related. Both
side-channels and covert channels are communication channels between two
entities in a system, where the entities are not supposed to be allowed
to communicate that way.

A \textbf{covert channel}\index{covert channel} is such a channel where
both entities intend to communicate through the channel. A
\textbf{side-channel}\index{side-channel} is a such a channel where one
end is the victim of an attack using the channel.

In other words, the difference between a covert channel and a
side-channel is whether both entities intend to communicate, in which
case we talk about a covert channel. If one entity does not intend to
communicate, but the other entity nonetheless extracts some data from
the first, it is called a side-channel attack. The entity not intending
to communicate, and hence being attacked, is called the
victim\index{victim}.

The rest of this chapter mostly describes a variety of common covert
channel mechanisms. It does not aim to differentiate much on whether
both ends intend to cooperate on the communication, or whether one end
is a victim under attack of the other end.

In the next few sections we'll explore a common few channels that can be
used as covert channels.

\hypertarget{cache-covert-channels}{%
\section{Cache covert channels}\label{cache-covert-channels}}

\href{https://en.wikipedia.org/wiki/Cache_(computing)}{Caches}\index{cache}
are used in almost every computing system. They are small and much
faster memories than the main memory. They aim to automatically keep
frequently used data accessed by programs, so that average memory access
time improves. Various techniques exist where a covert communication can
happen between processes that share a cache, without the processes
having rights to read or write to the same memory locations. To
understand how these techniques work, one needs to understand typical
organization and operation of a cache.

\hypertarget{typical-cpu-cache-architecture}{%
\subsection{Typical CPU cache
architecture}\label{typical-cpu-cache-architecture}}

There is a wide variety in
\href{https://en.wikipedia.org/wiki/CPU_cache}{CPU cache
micro-architecture} details, but the main characteristics that are
important to set up a covert channel tend to be similar across most
popular implementations.

Caches are small and much faster memories than the main memory that aim
to keep a copy of the data at the most frequently accessed main memory
addresses. The set of addresses that are used most frequently changes
quickly over time as a program executes. Therefore, the addresses that
are present in CPU caches also evolve quickly over time. The content of
the cache may change with every executed read or write instruction.

On every read and write instruction, the cache micro-architecture looks
up if the data for the requested address happens to be present in the
cache. If it is, the CPU can continue executing quickly; if not,
dependent operations will have to wait until the data returns from the
much slower main memory. A typical access time is 3 to 5 CPU cycles for
the fastest cache on a CPU versus hundreds of cycles for a main memory
access.\index{memory access time}

Most systems have multiple levels of cache\index{multi-level cache},
each with a different trade-off between cache size\index{cache size} and
access time\index{cache access time}. Some typical characteristics might
be:

\begin{itemize}
\tightlist
\item
  L1 (level 1) cache, 32kB in size, with an access time of 4 cycles.
\item
  L2 cache, 256Kb in size, with an access time of 10 cycles.
\item
  L3 cache, 16MB in size, with an access time of 40 cycles.
\item
  Main memory, gigabytes in size, with an access time of more than 100
  cycles.
\end{itemize}

\begin{fignos:no-prefix-figure-caption}

\begin{figure}
\centering
\includegraphics[width=0.4\textwidth,height=\textheight]{img/CacheLevels.pdf}
\caption{Illustration of cache levels in a typical system}
\end{figure}

\end{fignos:no-prefix-figure-caption}

If data is not already present in a cache layer, it is typically stored
there after it has been fetched from a slower cache level or main
memory. This is often a good decision to make as there's a high
likelihood the same address will be accessed by the program soon after.
This high likelihood is known as the
\href{https://en.wikipedia.org/wiki/Locality_of_reference}{principle of
locality}\index{principle
of locality}\index{locality of reference}.

Data is stored and transferred between cache levels in blocks of aligned
memory. Such a block is called a cache block\index{cache block} or cache
line\index{cache line}. Typical sizes are 32, 64 or 128 bytes per cache
line.

When data that wasn't previously in the cache needs to be stored in the
cache, most of the time, room has to be made for it by removing, or
evicting\index{cache eviction}, some other address/data from it. How
that choice gets made is decided by the
\href{https://en.wikipedia.org/wiki/Cache_replacement_policies}{cache
replacement policy}\index{cache
replacement policy}. Popular replacement algorithms are Least Recently
Used (LRU)\index{LRU replacement policy},
Random\index{random replacement policy} and
pseudo-LRU\index{pseudo-LRU replacement policy}. As the names suggest,
LRU evicts the cache line that is least recently used; random picks a
random cache line; and pseudo-LRU approximates choosing the least
recently used line.

If an address can be stored in all locations available in the cache, the
cache is fully-associative\index{fully-associative cache}. Most caches
are however not fully-associative, as it's too costly to implement.
Instead, most caches are set-associative\index{set-associative cache}.
In an N-way set-associative cache, a specific main memory address can
only be stored in one of N cache locations. For example, if an address
can potentially be stored in one of 2 locations, the cache is said to be
2-way set-associative. If it can be stored in one of 4 locations, it's
called 4-way set-associative, and so on. When an address can only be
stored in one location in the cache, it is said to be
direct-mapped\index{direct-mapped cache}, rather than 1-way
set-associative. Typical organizations are direct-mapped, 2-way, 4-way,
8-way, 16-way or 32-way set-associative.

\missingcontent{Explain indexing mechanism used; from address bits to index in cache.}

\missingcontent{Also explain cache coherency \index{cache coherency}?}
\missingcontent{Also say something about TLBs and prefetching?}

\hypertarget{general-operation-of-cache-covert-channels}{%
\subsection{General operation of cache covert
channels}\label{general-operation-of-cache-covert-channels}}

\hypertarget{timing-covert-channels}{%
\section{Timing covert channels}\label{timing-covert-channels}}

\hypertarget{resource-contention-channels}{%
\section{Resource contention
channels}\label{resource-contention-channels}}

\hypertarget{channels-making-use-of-aliasing-in-branch-predictors-and-other-predictors}{%
\section{Channels making use of aliasing in branch predictors and other
predictors}\label{channels-making-use-of-aliasing-in-branch-predictors-and-other-predictors}}

\missingcontent{Should we also discuss more "covert" channels here such as power analysis, etc?}

\hypertarget{physical-access-side-channel-attacks}{%
\chapter{Physical access side-channel
attacks}\label{physical-access-side-channel-attacks}}

\missingcontent{Write chapter on physical access side-channel attacks.}

\hypertarget{remote-access-side-channel-attacks}{%
\chapter{Remote access side-channel
attacks}\label{remote-access-side-channel-attacks}}

This chapter covers side-channel attacks for which the attacker does not
need physical access to the hardware.

\hypertarget{timing-attacks}{%
\section{Timing attacks}\label{timing-attacks}}

An implementation of a cryptographic algorithm can leak information
about the data it processes if its run time is influenced by the value
of the processed data. Attacks making use of this are called timing
attacks\index{timing
attacks}.

The main mitigation against such attacks consists of carefully
implementing the algorithm such that the execution time remains
independent of the processed data. This can be done by making sure that
both:

\begin{enumerate}
\def\labelenumi{\alph{enumi})}
\item
  The control flow, i.e.~the trace of instructions executed, does not
  change depending on the processed data. This guarantees that every
  time the algorithm runs, exactly the same sequence of instructions is
  executed, independent of the processed data.
\item
  The instructions used to implement the algorithm are from the subset
  of instructions for which the execution time is known to not depend on
  the data values it processes.

  For example, in the Arm architecture, the Armv8.4-A
  \href{https://developer.arm.com/documentation/ddi0595/2021-06/AArch64-Registers/DIT--Data-Independent-Timing}{DIT
  extension} guarantees that execution time is data-independent for a
  subset of the AArch64 instructions.

  By ensuring that the extension is enabled and only instructions in the
  subset are used, data-independent execution time is guaranteed.
\end{enumerate}

At the moment, we do not know of a compiler implementation that actively
helps to guarantee both (a) and (b). A great reference giving practical
advice on how to achieve (a), (b) and more security hardening properties
specific for cryptographic kernels is found in (Pornin
\protect\hyperlink{ref-Pornin2018}{2018}).

As discussed in (Pornin \protect\hyperlink{ref-Pornin2018}{2018}), when
implementing cryptographic algorithms, you also need to keep cache
side-channel attacks in mind, which are discussed in the
\protect\hyperlink{cache-side-channel-attacks}{section on cache
side-channel attacks}.

\hypertarget{cache-side-channel-attacks}{%
\section{Cache side-channel attacks}\label{cache-side-channel-attacks}}

\missingcontent{Write section on cache side-channel attacks. See
\href{https://github.com/llsoftsec/llsoftsecbook/pull/24\#issuecomment-930266031}{the first comment on PR24}
for suggestions of what this should contain.}

\hypertarget{supply-chain-attacks}{%
\chapter{Supply chain attacks}\label{supply-chain-attacks}}

A software \emph{supply chain attack} occurs when an attacker interferes
with the software development or distribution processes with the
intention to impact users of that software.

Supply chain attacks and their possible mitigations are not specific to
compilers. However, compilers are an attractive target for attack
because they are widely deployed to developers, in continuous
integration systems and as JITs. Also, an infected compiler has the
possibility to make a much larger impact if it can silently spread the
infection to other software created with or run using it.

This chapter explores the history of supply chain attacks that involve
compilers and what can be done to prevent them.

\hypertarget{history-of-supply-chain-attacks}{%
\section{History of supply chain
attacks}\label{history-of-supply-chain-attacks}}

As far back as 1974 Karger \& Schell theorized about an attack on the
Multics operating system via the PL/I compiler (Paul A. and Roger R.
\protect\hyperlink{ref-Karger1974}{1974}). In this attack, a trap door
is inserted into the compiler, which then injects malicious code into
generated object code. Furthermore, the trap door could be designed to
reinsert itself into the compiler binary so that future compilers are
silently infected without needing changes to their source code. This
attack method was subsequently popularised by Ken Thompson in his 1984
ACM Turing Award acceptance speech \emph{Reflections on Trusting Trust}
(Thompson \protect\hyperlink{ref-Thompson1984}{1984}).

If these cases seem far-fetched then consider that there have been
several real examples of supply chain attacks on development tools.

Induc is a family of viruses that infects a pre-compiled library in the
Delphi toolchain with malicious code (Gostev
\protect\hyperlink{ref-Gostev2009}{2009}). When Delphi compiles a
project the malicious library is included into the resulting executable,
thus enabling the virus to spread. The virus was first detected in 2009
and was circulating undetected for at least a year beforehand. Several
popular applications are known to have been infected, including a chat
client and a media player. Overall, in excess of a hundred thousand
infected computers were detected world-wide by anti-virus solutions.

XcodeGhost is the name given to malware first detected in 2015 that
infected thousands of iOS applications (Cox
\protect\hyperlink{ref-Cox2015}{2015}). The source of the infection was
tracked down to a trojanized version of Xcode tools. The malware exists
in an extra object file within the Xcode tools and is silently linked
into each application as it is built. File sharing sites were used to
spread the trojanized Xcode tools to unwitting developers.

A trojanized linker was found to be involved in a supply chain attack
discovered in 2017 named ShadowPad (Greenberg
\protect\hyperlink{ref-Greenberg2019}{2019}). Some instances of the
attack were perpetrated using a trojanized Visual Studio linker that
silently incorporates a malicious library into applications as they are
built. Related attacks named CCleaner and ShadowHammer used the same
approach of a trojanized linker to infect built applications. Infected
applications from these attacks were distributed to millions of users
world-wide.

These cases highlight that attacks on compilers, and especially linkers
and libraries, are a viable route to silently infect many other
applications, and there is no doubt that there will be more such attacks
in the future. Let us now explore what we can do about these.

\missingcontent{Explain how these vulnerabilities arise and how to mitigate them.}

\hypertarget{other-security-topics-relevant-for-compiler-developers}{%
\chapter{Other security topics relevant for compiler
developers}\label{other-security-topics-relevant-for-compiler-developers}}

\missingcontent{Write chapter with other security topics.}

\missingcontent{Write section on securely clearing memory in C/C++ and undefined behaviour.}

\hypertarget{appendix-contribution-guidelines}{%
\chapter*{Appendix: contribution
guidelines}\label{appendix-contribution-guidelines}}
\addcontentsline{toc}{chapter}{Appendix: contribution guidelines}

\missingcontent{Write chapter on contribution guidelines.
 These should include at least: project locaton on github; how to create pull requests/issues.
 Where do we discuss - mailing list? Grammar and writing style guidelines.
 How to use todos and index.}

\printindex

\listoftodos

\hypertarget{references}{%
\chapter*{References}\label{references}}
\addcontentsline{toc}{chapter}{References}

\hypertarget{refs}{}
\leavevmode\hypertarget{ref-AlephOne1996}{}%
Aleph One. 1996. ``Smashing the Stack for Fun and Profit.'' 1996.
\url{http://www.phrack.org/issues/49/14.html\#article}.

\leavevmode\hypertarget{ref-Beer2020}{}%
Beer, Ian. 2020. ``An iOS Zero-Click Radio Proximity Exploit Odyssey.''
2020.
\url{https://googleprojectzero.blogspot.com/2020/12/an-ios-zero-click-radio-proximity.html}.

\leavevmode\hypertarget{ref-Beer2021}{}%
Beer, Ian, and Samuel Groß. 2021. ``A Deep Dive into an Nso Zero-Click
iMessage Exploit: Remote Code Execution.'' 2021.
\url{https://googleprojectzero.blogspot.com/2021/12/a-deep-dive-into-nso-zero-click.html}.

\leavevmode\hypertarget{ref-Cox2015}{}%
Cox, Joseph. 2015. ``Hack Brief: Malware Sneaks into the Chinese iOS App
Store.'' \emph{WIRED}.
\url{https://www.wired.com/2015/09/hack-brief-malware-sneaks-chinese-ios-app-store/}.

\leavevmode\hypertarget{ref-Dullien2020}{}%
Dullien, Thomas. 2020. ``Weird Machines, Exploitability, and Provable
Unexploitability.'' \emph{IEEE Transactions on Emerging Topics in
Computing} 8 (2): 391--403.
\url{https://doi.org/10.1109/TETC.2017.2785300}.

\leavevmode\hypertarget{ref-Gostev2009}{}%
Gostev, Alexander. 2009. ``A Short History of Induc.'' 2009.
\url{https://securelist.com/a-short-history-of-induc/30555/}.

\leavevmode\hypertarget{ref-Greenberg2019}{}%
Greenberg, Andy. 2019. ``Supply Chain Hackers Snuck Malware into
Videogames.'' \emph{WIRED}.
\url{https://www.wired.com/story/supply-chain-hackers-videogames-asus-ccleaner/}.

\leavevmode\hypertarget{ref-Grouxdf2020}{}%
Groß, Samuel. 2020. ``JITSploitation I: A Jit Bug.'' 2020.
\url{https://googleprojectzero.blogspot.com/2020/09/jitsploitation-one.html}.

\leavevmode\hypertarget{ref-Hicks2014}{}%
Hicks, Michael. 2014. ``What Is Memory Safety?'' 2014.
\url{http://www.pl-enthusiast.net/2014/07/21/memory-safety/}.

\leavevmode\hypertarget{ref-Hu2016}{}%
Hu, Hong, Shweta Shinde, Sendroiu Adrian, Zheng Leong Chua, Prateek
Saxena, and Zhenkai Liang. 2016. ``Data-Oriented Programming: On the
Expressiveness of Non-Control Data Attacks.'' In \emph{2016 Ieee
Symposium on Security and Privacy (Sp)}, 969--86.
\url{https://doi.org/10.1109/SP.2016.62}.

\leavevmode\hypertarget{ref-Miller2012}{}%
Miller, Matt. n.d. ``Modeling the Exploitation and Mitigation of Memory
Safety Vulnerabilities.''
\href{https://2012.ruxconbreakpoint.com/}{Breakpoint 2012}.
\url{https://github.com/Microsoft/MSRC-Security-Research/blob/master/presentations/2012_10_Breakpoint/BreakPoint2012_Miller_Modeling_the_exploitation_and_mitigation_of_memory_safety_vulnerabilities.pdf}.

\leavevmode\hypertarget{ref-Karger1974}{}%
Paul A., Karger, and Schell Roger R. 1974. ``MULTICS Security
Evaluation: VULNERABILITY Analysis,'' 52.
\url{https://csrc.nist.gov/csrc/media/publications/conference-paper/1998/10/08/proceedings-of-the-21st-nissc-1998/documents/early-cs-papers/karg74.pdf}.

\leavevmode\hypertarget{ref-Pornin2018}{}%
Pornin, Thomas. 2018. ``Why Constant-Time Crypto?'' 2018.
\url{https://www.bearssl.org/constanttime.html}.

\leavevmode\hypertarget{ref-Shacham2007}{}%
Shacham, Hovav. 2007. ``The Geometry of Innocent Flesh on the Bone:
Return-into-Libc Without Function Calls (on the X86).'' In
\emph{Proceedings of the 14th Acm Conference on Computer and
Communications Security}, 552--61. CCS '07. New York, NY, USA:
Association for Computing Machinery.
\url{https://doi.org/10.1145/1315245.1315313}.

\leavevmode\hypertarget{ref-Solar1997}{}%
Solar Designer. 1997. ``Getting Around Non-Executable Stack (and Fix).''
1997. \url{https://seclists.org/bugtraq/1997/Aug/63}.

\leavevmode\hypertarget{ref-Thompson1984}{}%
Thompson, Ken. 1984. ``Reflections on Trusting Trust.''
\url{https://www.cs.cmu.edu/~rdriley/487/papers/Thompson_1984_ReflectionsonTrustingTrust.pdf}.

\end{document}
