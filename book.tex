\PassOptionsToPackage{unicode=true}{hyperref} % options for packages loaded elsewhere
\PassOptionsToPackage{hyphens}{url}
\PassOptionsToPackage{dvipsnames,svgnames*,x11names*}{xcolor}
%
\documentclass[a4paper,]{report}
\usepackage{lmodern}
\usepackage{amssymb,amsmath}
\usepackage{ifxetex,ifluatex}
\usepackage{fixltx2e} % provides \textsubscript
\ifnum 0\ifxetex 1\fi\ifluatex 1\fi=0 % if pdftex
  \usepackage[T1]{fontenc}
  \usepackage[utf8]{inputenc}
  \usepackage{textcomp} % provides euro and other symbols
\else % if luatex or xelatex
  \usepackage{unicode-math}
  \defaultfontfeatures{Ligatures=TeX,Scale=MatchLowercase}
\fi
% use upquote if available, for straight quotes in verbatim environments
\IfFileExists{upquote.sty}{\usepackage{upquote}}{}
% use microtype if available
\IfFileExists{microtype.sty}{%
\usepackage[]{microtype}
\UseMicrotypeSet[protrusion]{basicmath} % disable protrusion for tt fonts
}{}
\IfFileExists{parskip.sty}{%
\usepackage{parskip}
}{% else
\setlength{\parindent}{0pt}
\setlength{\parskip}{6pt plus 2pt minus 1pt}
}
\usepackage{xcolor}
\usepackage{hyperref}
\hypersetup{
            pdftitle={Low Level Software Security for Compiler Developers},
            colorlinks=true,
            linkcolor=Maroon,
            filecolor=Maroon,
            citecolor=Blue,
            urlcolor=Blue,
            breaklinks=true}
\urlstyle{same}  % don't use monospace font for urls
\setlength{\emergencystretch}{3em}  % prevent overfull lines
\providecommand{\tightlist}{%
  \setlength{\itemsep}{0pt}\setlength{\parskip}{0pt}}
\setcounter{secnumdepth}{5}
% Redefines (sub)paragraphs to behave more like sections
\ifx\paragraph\undefined\else
\let\oldparagraph\paragraph
\renewcommand{\paragraph}[1]{\oldparagraph{#1}\mbox{}}
\fi
\ifx\subparagraph\undefined\else
\let\oldsubparagraph\subparagraph
\renewcommand{\subparagraph}[1]{\oldsubparagraph{#1}\mbox{}}
\fi

% set default figure placement to htbp
\makeatletter
\def\fps@figure{htbp}
\makeatother

\usepackage{makeidx}
\makeindex
\newcounter{TodoCounter}
\usepackage[backgroundcolor=white,linecolor=black]{todonotes}
\let\oldtodo\todo
\usepackage{bclogo}%  \bcpanchant
\renewcommand{\todo}[1]{
  \stepcounter{TodoCounter}
  \oldtodo[caption={\arabic{TodoCounter}. #1}]{\bcpanchant #1}
}
\newcommand{\missingcontent}[1]{
  \stepcounter{TodoCounter}
  \oldtodo[inline,caption={\arabic{TodoCounter}. #1}]{\bcpanchant \textit{#1}}
}

\title{Low Level Software Security for Compiler Developers}
\date{}

\begin{document}
\maketitle

\clearpage

\vspace*{\fill}
%\includegraphics[height=2cm]{example-image}
This work is licensed under the Creative Commons Attribution 4.0 International
License. To view a copy of this license, visit
http://creativecommons.org/licenses/by/4.0/ or send a letter to Creative
Commons, PO Box 1866, Mountain View, CA 94042, USA.

  © 2021 Arm Limited
  \href{mailto:kristof.beyls@arm.com}{\nolinkurl{kristof.beyls@arm.com}}\\
\clearpage

{
\hypersetup{linkcolor=}
\setcounter{tocdepth}{2}
\tableofcontents
}
\hypertarget{introduction}{%
\chapter{Introduction}\label{introduction}}

Compilers, assemblers and similar tools generate all the binary code
that processors execute. It is no surprise then that for security
analysis and hardening relevant for binary code, these tools have a
major role to play. Often the only practical way to protect all binaries
with a particular security hardening method is to let the compiler adapt
its automatic code generation.

With software security becoming even more important in recent years, it
is no surprise to see an ever increasing variety of security hardening
features and mitigations against vulnerabilities implemented in
compilers.

Indeed, compared to a few decades ago, today's compiler developer is
much more likely to work on security features, at least some of their
time.

Furthermore, with the ever-expanding range of techniques implemented, it
has become very hard to gain a basic understanding of all security
features implemented in typical compilers.

This poses a practical problem: compiler developers must be able to work
on security hardening features, yet it is hard to gain a good basic
understanding of such compiler features.

This book aims to help developers of code generation tools such as JITs,
compilers, linkers and assemblers to overcome this.

There is a lot of material that can be found explaining individual
vulnerabilities or attack vectors. There are also lots of presentations
explaining specific exploits. But there seems to be a limited set of
material that gives a structured overview of all vulnerabilities and
exploits for which a code generator could play a role in protecting
against them.

This book aims to provide such a structured, broad overview. It does not
necessarily go into full details. Instead it aims to give a thorough
description of all relevant high-level aspects of attacks,
vulnerabilities, mitigations and hardening techniques. For further
details, this book provides pointers to material with more details on
specific techniques.

The purpose of this book is to serve as a guide to every compiler
developer that needs to learn about software security relevant to
compilers. Even though the focus is on compiler developers, we expect
that this book will also be useful to other people working on low-level
software.

\hypertarget{how-this-book-is-created}{%
\section{How this book is created}\label{how-this-book-is-created}}

The idea for this book emerged out of a frustration of not finding a
good overview on this topic. Kristof Beyls and Georgia Kouveli, both
compiler engineers working on security features, wished a book like this
would exist. After not finding such a book, they decided to try and
write one themselves. They immediately realized that they do not have
all necessary expertise themselves to complete such a daunting task. So
they decided to try and create this book in an open source style,
seeking contributions from many experts.

As you read this, the book remains unfinished. This book may well never
be finished, as new vulnerabilities continue to be discovered regularly.
Our hope is that developing the book as an open source project will
allow for it to continue to evolve and improve. The open source
development process of this book increases the likelihood that it
remains relevant as new vulnerabilities and mitigations emerge.

Kristof and Georgia, the initial authors, are far from experts on all
possible vulnerabilities. So what is the plan to get high quality
content to cover all relevant topics? It is two-fold.

First, by studying specific topics, they hope to gain enough knowledge
to write up a good summary for this book.

Second, they very much invite and welcome contributions. If you're
interested in potentially contributing content, please go to the home
location for the open source project at
\url{https://github.com/llsoftsec/llsoftsecbook}.

As a reader, you can also contribute to making this book better. We
highly encourage feedback, both positive and constructive criticisms. We
prefer feedback to be received through
\url{https://github.com/llsoftsec/llsoftsecbook}.

\missingcontent{Add section describing the structure of the rest of the book.}

\hypertarget{memory-vulnerability-based-attacks-and-mitigations}{%
\chapter{Memory vulnerability based attacks and
mitigations}\label{memory-vulnerability-based-attacks-and-mitigations}}

\missingcontent{Write chapter on memory vulnerabilities and mitigation.}

\hypertarget{physical-access-side-channel-attacks}{%
\chapter{Physical access side-channel
attacks}\label{physical-access-side-channel-attacks}}

\missingcontent{Write chapter on physical access side-channel attacks.}

\hypertarget{remote-access-side-channel-attacks}{%
\chapter{Remote access side-channel
attacks}\label{remote-access-side-channel-attacks}}

This chapter covers side-channel attacks for which the attacker does not
need physical access to the hardware.\todo{Define side-channel better.}

\hypertarget{timing-attacks}{%
\section{Timing attacks}\label{timing-attacks}}

An implementation of a cryptographic algorithm can leak information
about the data it processes if its run time is influenced by the value
of the processed data. Attacks making use of this are called timing
attacks\index{timing
attacks}.

The main mitigation against such attacks consists of carefully
implementing the algorithm such that the execution time remains
independent of the processed data. This can be done by making sure that
both:

\begin{enumerate}
\def\labelenumi{\alph{enumi})}
\item
  The control flow, i.e.~the trace of instructions executed, does not
  change depending on the processed data. This guarantees that every
  time the algorithm runs, exactly the same sequence of instructions is
  executed, independent of the processed data.
\item
  The instructions used to implement the algorithm are from the subset
  of instructions for which the execution time is known to not depend on
  the data values it processes.

  For example, in the Arm architecture, the Armv8.4-A
  \href{https://developer.arm.com/documentation/ddi0595/2021-06/AArch64-Registers/DIT--Data-Independent-Timing}{DIT
  extension} guarantees that execution time is data-independent for a
  subset of the AArch64 instructions.

  By ensuring that the extension is enabled and only instructions in the
  subset are used, data-independent execution time is guaranteed.
\end{enumerate}

At the moment, we do not know of a compiler implementation that actively
helps to guarantee both (a) and (b). A great reference giving practical
advice on how to achieve (a), (b) and more security hardening properties
specific for cryptographic kernels is found in (Pornin
\protect\hyperlink{ref-Pornin2018}{2018}).

As discussed in (Pornin \protect\hyperlink{ref-Pornin2018}{2018}), when
implementing cryptographic algorithms, you also need to keep cache
side-channel attacks in mind, which are discussed in the
\protect\hyperlink{cache-side-channel-attacks}{section on cache
side-channel attacks}.

\hypertarget{cache-side-channel-attacks}{%
\section{Cache side-channel attacks}\label{cache-side-channel-attacks}}

\missingcontent{Write section on cache side-channel attacks. See
\href{https://github.com/llsoftsec/llsoftsecbook/pull/24\#issuecomment-930266031}{the first comment on PR24}
for suggestions of what this should contain.}

\hypertarget{other-security-topics-relevant-for-compiler-developers}{%
\chapter{Other security topics relevant for compiler
developers}\label{other-security-topics-relevant-for-compiler-developers}}

\missingcontent{Write chapter with other security topics.}

\missingcontent{Write section on securely clearing memory in C/C++ and undefined behaviour.}

\hypertarget{appendix-contribution-guidelines}{%
\chapter*{Appendix: contribution
guidelines}\label{appendix-contribution-guidelines}}
\addcontentsline{toc}{chapter}{Appendix: contribution guidelines}

\missingcontent{Write chapter on contribution guidelines.
 These should include at least: project locaton on github; how to create pull requests/issues.
 Where do we discuss - mailing list? Grammar and writing style guidelines.
 How to use todos and index.}

\printindex

\listoftodos

\hypertarget{references}{%
\chapter*{References}\label{references}}
\addcontentsline{toc}{chapter}{References}

\hypertarget{refs}{}
\leavevmode\hypertarget{ref-Pornin2018}{}%
Pornin, Thomas. 2018. ``Why Constant-Time Crypto?'' 2018.
\url{https://www.bearssl.org/constanttime.html}.

\end{document}
